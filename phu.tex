%!TeX program = lualatex
%!TeX root = main.tex


\section{Cây Đỏ-Đen}

\subsection{Giới thiệu}

\begin{frame}
	\frametitle{Thành phần của nút trong cây đỏ-đen}
	Nút trong cây đỏ-đen gồm các thuộc tính là \begin{description}
		\item[color] Màu của nút đó.
		\item[key] Giá trị được lưu.
		\item[left] Nút con bên trái.
		\item[right] Nút con bên phải.
		\item[parent] Nút cha.
	\end{description}
\end{frame}

\begin{frame}
	\frametitle{Định nghĩa cây đỏ-đen}
	Cây đỏ-đen là một cây nhị phân tìm kiếm thỏa mãn các tính chất sau: \begin{enumerate}
		\item Mỗi nút trong cây phải có màu đen hoặc đỏ.
		\item Nút gốc phải có màu đen.
		\item Nút lá phải có màu đen.
		\item Một nút có màu đỏ thì tất cả nút con nó phải màu đen.
		\item Mọi đường đi từ một nút bất kì trong cây đến các nút hậu duệ lá đều phải có số lượng nút đen bằng nhau.
	\end{enumerate}
\end{frame}

\begin{frame}
	\frametitle{Đặc điểm của đây đỏ-đen}
	\begin{itemize}
		\item Mọi đường đi từ nút gốc đến nút lá có số nút chênh nhau không quá 2 lần.
		\item Mọi cây con của cây đỏ-đen cũng là cây đỏ-đen\only<2->{ \alert<2>{nếu đổi màu nút gốc thành đen}}.
		\item Tính chất cuối có thể thay \alert{nút bất kì} thành \alert{nút gốc}.
	\end{itemize}
\end{frame}

\begin{frame}
	\frametitle{Ví dụ về cây đỏ-đen}
		\centering
		\begin{forest}
			for tree={s sep= 3mm},
			rb-tree
			[26
				[17,red-node
					[14
						[10,red-node
							[7
								[3,red-node
									[,nil-node]
									[,nil-node]
								]
								[,nil-node]
							]
							[12
								[,nil-node]
								[,nil-node]
							]
						]
						[16
							[15,red-node
								[,nil-node]
								[,nil-node]
							]
							[,nil-node]
						]
					]
					[21
						[19
							[,nil-node]
							[20,red-node
								[,nil-node]
								[,nil-node]
							]
						]
						[23
							[,nil-node]
							[,nil-node]
						]
					]
				]
				[41
					[30,red-node
						[28
							[,nil-node]
							[,nil-node]
						]
						[38
							[35,red-node
								[,nil-node]
								[,nil-node]
							]
							[39,red-node
								[,nil-node]
								[,nil-node]
							]
						]
					]
					[47
						[,nil-node]
						[,nil-node]
					]
				]
			]
		\end{forest}
\end{frame}

\subsection{Các thao tác trên cây}
\begin{frame}
\frametitle{Phép quay}
\framesubtitle{Minh họa}
	\centering
	\begin{twotrees}
	{
		\begin{forest}
			tree,
			for tree={fit=tight,font={\scriptsize},s sep=2em,l=5mm}
			[,empty-node
				[y,rand-node
					[x,rand-node
						[$\alpha$,non-decorate-node]
						[$\beta$,non-decorate-node]
					]
					[$\gamma$,non-decorate-node]
				]
			]
		\end{forest}
	}
	{
		\begin{forest}
			tree,
			for tree={fit=tight,font=\scriptsize,s sep=2em,l=5mm}
			[,empty-node
				[x,rand-node
					[$\gamma$,non-decorate-node]
					[y,rand-node
						[$\beta$,non-decorate-node]
						[$\gamma$,non-decorate-node]
					]
				]
			]
		\end{forest}
	}
	{2cm}
	{Xoay phải/Xoay trái}
	\end{twotrees}
\end{frame}

\begin{frame}
	\frametitle{Phép chèn trong cây đỏ-đen}
	\framesubtitle{Thuật toán}
	
	\begin{block}{Thuật toán chèn nút cây đỏ-đen}
		\begin{enumerate}
			\item Thực hiện phép chèn của BST
			\item Tô màu nút mới màu đỏ
			\item Thực hiện sửa lỗi tại nút mới
		\end{enumerate}
	\end{block}
\end{frame}

\begin{frame}
	\frametitle{Lỗi phép chèn trong cây-đỏ}
	\framesubtitle{Trường hợp 1}
	\centering
	\begin{twotrees}
	{
		\begin{forest}
			rb-tree,
			for tree={fit=band,l=0pt,s sep=1em}
			[,empty-node
				[C
					[B,red-node
						[A,red-node,label=right:x
							[$\alpha$,non-decorate-node]
							[$\beta$,non-decorate-node]
						]
						[$\gamma$,non-decorate-node]
					]
					[$\delta$,non-decorate-node,label=right:u,name=delta]
				]
			]
			\node (B) at ([xshift=0.5cm,yshift=-0.5cm] delta) {đen}; % node ở cuối mũi tên
			\draw[->, bend right=20] (delta.south) to (B.west);
		\end{forest}
	}
	{
		\begin{forest}
			rb-tree,
			for tree={fit=band,l=0pt,s sep=1em}
			[,empty-node
				[B
					[A,red-node,label=left:x
						[$\alpha$,non-decorate-node]
						[$\beta$,non-decorate-node]
					]
					[C,red-node
						[$\gamma$,non-decorate-node]
						[$\delta$,non-decorate-node]
					]
				]
			]
		\end{forest}
	}
	{2cm}
	{/}
	\end{twotrees}
\end{frame}

\begin{frame}
	\frametitle{Lỗi phép chèn trong cây đỏ-đen}
	\framesubtitle{Trường hợp 2}
	\centering
	\begin{twotrees}
	{
		\begin{forest}
			rb-tree,
			for tree={fit=band,l=0pt,s sep=1em}
			[,empty-node
				[C
					[A,red-node
						[$\alpha$,non-decorate-node]
						[B,red-node,label=right:x
							[$\beta$,non-decorate-node]
							[$\gamma$,non-decorate-node]
						]
					]
					[$\delta$,non-decorate-node,label=right:u,name=delta]
				]
			]
			\node (B) at ([xshift=0.5cm,yshift=-0.5cm] delta) {đen}; % node ở cuối mũi tên
			\draw[->, bend right=20] (delta.south) to (B.west);
		\end{forest}
	}
	{
		\begin{forest}
			rb-tree,
			for tree={fit=band,l=0pt,s sep=1em}
			[,empty-node
				[C
					[B,red-node,s sep=2.2em
						[A,red-node,label=right:{new x}
							[$\alpha$,non-decorate-node]
							[$\beta$,non-decorate-node]
						]
						[$\gamma$,non-decorate-node]
					]
					[$\delta$,non-decorate-node,label=right:u]
				]
			]
		\end{forest}
	}
	{2cm}
	{/}
	\end{twotrees}
\end{frame}

\begin{frame}
	\frametitle{Lỗi phép chèn trong cây đỏ-đen}
	\framesubtitle{Trường hợp 3}
	\centering
	\begin{twotrees}
	{
		\begin{forest}
			rb-tree,
			for tree={fit=band,l=0pt,s sep=1em}
			[,empty-node
				[C
					[A,red-node
						[$\alpha$,non-decorate-node]
						[B,red-node,label=right:x
							[$\beta$,non-decorate-node]
							[$\gamma$,non-decorate-node]
						]
					]
					[D,red-node,label=right:u
						[$\delta$,non-decorate-node]
						[$\varepsilon$,non-decorate-node]
					]
				]
			]
		\end{forest}
	}
	{
		\begin{forest}
			rb-tree,
			for tree={fit=band,l=0pt,s sep=1em}
			[,empty-node
				[C,red-node,label=right:{new x}
					[A
						[$\alpha$,non-decorate-node]
						[B,red-node
							[$\beta$,non-decorate-node]
							[$\gamma$,non-decorate-node]
						]
					]
					[D
						[$\delta$,non-decorate-node]
						[$\varepsilon$,non-decorate-node]
					]
				]
			]
		\end{forest}
	}
	{2cm}
	{/}
	\end{twotrees}
\end{frame}

\begin{frame}
	\frametitle{Lỗi phép chèn  trong cây đỏ-đen}
	\framesubtitle{Trường hợp 3 (tiếp theo)}
	\centering
	\begin{twotrees}
	{
		\begin{forest}
			rb-tree,
			for tree={fit=band,l=0pt,s sep=1em}
			[,empty-node
				[C
					[A,red-node
						[B,red-node,label=right:x
							[$\alpha$,non-decorate-node]
							[$\beta$,non-decorate-node]
						]
						[$\gamma$,non-decorate-node]
					]
					[D,red-node,label=right:u
						[$\delta$,non-decorate-node]
						[$\varepsilon$,non-decorate-node]
					]
				]
			]
		\end{forest}
	}
	{
		\begin{forest}
			rb-tree,
			for tree={fit=band,l=0pt,s sep=1em}
			[,empty-node
				[C,red-node,label=right:{new x}
					[A
						[B,red-node
							[$\alpha$,non-decorate-node]
							[$\beta$,non-decorate-node]
						]
						[$\gamma$,non-decorate-node]
					]
					[D
						[$\delta$,non-decorate-node]
						[$\varepsilon$,non-decorate-node]
					]
				]
			]
		\end{forest}
	}
	{2cm}
	{/}
	\end{twotrees}
\end{frame}

\begin{frame}
	\frametitle{Đánh giá sửa lỗi chèn trong cây đỏ-đen}
	Trong những trường hợp nào thì thì các thao tác sửa lỗi chèn thực hiện trong $O(1)$?
	\begin{itemize}
	\item Trường hợp 1
	\item Trường hợp 2
	\end{itemize}
\end{frame}

\begin{frame}
	\frametitle{Phép xóa trong cây đỏ-đen}
	\framesubtitle{Thuật toán}
	\begin{block}{Thuật toán xóa nút cây đỏ-đen}
		\begin{enumerate}
			\item Thực hiện phép xóa của BST (khi thay thế nút, không thay thế màu)
			\item Thực hiện sửa lỗi tại vị trí bị ảnh hưởng (vị trí bị xóa thực sự)
		\end{enumerate}
	\end{block}
\end{frame}


\begin{frame}
	\frametitle{Lỗi phép xóa trong cây đỏ-đen}
	\framesubtitle{Trường hợp 1}
	\centering
	\begin{twotrees}
	{
		\begin{forest}
			rb-tree,for tree={fit=band,l=0pt,s sep=0.6em}
			[,empty-node
				[B,rand-node,label=right:p
					[A,label=right:x
						[$\alpha$,non-decorate-node]
						[$\beta$,non-decorate-node]
					]
					[D,label=right:s
						[C,rand-node,label=right:{n}
							[$\gamma$,non-decorate-node]
							[$\delta$,non-decorate-node]
						]
						[E,red-node
							[$\varepsilon$,non-decorate-node]
							[$\zeta$,non-decorate-node]
						]
					]
				]	
			]
		\end{forest}
	}
	{
		\begin{forest}
			rb-tree,for tree={fit=band,l=0pt,s sep=0.6em}
			[,empty-node
				[D,rand-node
					[B
						[A
							[$\alpha$,non-decorate-node]
							[$\beta$,non-decorate-node]
						]
						[C,rand-node,label=right:{n}
							[$\gamma$,non-decorate-node]
							[$\delta$,non-decorate-node]
						]
					]
					[E,
						[$\varepsilon$,non-decorate-node]
						[$\zeta$,non-decorate-node]
					]
				]
			]
		\end{forest}
	}
	{2cm}
	{/}
	\end{twotrees}
\end{frame}


\begin{frame}
	\frametitle{Lỗi phép xóa trong cây đỏ-đen}
	\framesubtitle{Trường hợp 2}
	\centering
	\begin{twotrees}
	{
		\begin{forest}
			rb-tree,for tree={fit=band,l=0pt,s sep=0.6em}
			[,empty-node
				[B,rand-node,label=right:p
					[A,label=right:x
						[$\alpha$,non-decorate-node]
						[$\beta$,non-decorate-node]
					]
					[D,label=right:s
						[C,red-node,label=right:n
							[$\gamma$,non-decorate-node]
							[$\delta$,non-decorate-node]
						]
						[E
							[$\varepsilon$,non-decorate-node]
							[$\zeta$,non-decorate-node]
						]
					]
				]	
			]
		\end{forest}
	}
	{
		\begin{forest}
			rb-tree,for tree={fit=band,l=0pt,s sep=0.6em}
			[,empty-node
				[B,rand-node,label=right:p
					[A,label=right:x
						[$\alpha$,non-decorate-node]
						[$\beta$,non-decorate-node]
					]
					[C,label=right:{new s}
						[$\gamma$,non-decorate-node]
						[D,red-node,label=right:{new n}
							[$\delta$,non-decorate-node]
							[E
								[$\varepsilon$,non-decorate-node]
								[$\zeta$,non-decorate-node]
							]
						]
					]
				]	
			]
		\end{forest}
	}
	{1.8cm}
	{/}
	\end{twotrees}
\end{frame}


\begin{frame}
	\frametitle{Lỗi phép xóa trong cây đỏ-đen}
	\framesubtitle{Trường hợp 3}
	\begin{twotrees}
	{
		\begin{forest}
			rb-tree,for tree={fit=band,l=0pt,s sep=0.6em}
			[,empty-node
				[B,rand-node,label=right:p
					[A,label=left:x
						[$\alpha$,non-decorate-node]
						[$\beta$,non-decorate-node]
					]
					[D,label=right:s
						[C
							[$\gamma$,non-decorate-node]
							[$\delta$,non-decorate-node]
						]
						[E
							[$\varepsilon$,non-decorate-node]
							[$\zeta$,non-decorate-node]
						]
					]
				]
			]
		\end{forest}
	}
	{
		\begin{forest}
			rb-tree,for tree={fit=band,l=0pt,s sep=0.6em}
			[,empty-node
				[B,rand-node,label=right:p,label=left:{new x}
					[A
						[$\alpha$,non-decorate-node]
						[$\beta$,non-decorate-node]
					]
					[D,red-node,label=right:s
						[C
							[$\gamma$,non-decorate-node]
							[$\delta$,non-decorate-node]
						]
						[E
							[$\varepsilon$,non-decorate-node]
							[$\zeta$,non-decorate-node]
						]
					]
				]
			]
		\end{forest}
	}
	{2cm}
	{/}
	\end{twotrees}
\end{frame}

\begin{frame}
	\frametitle{Lỗi phép xóa trong cây đỏ-đen}
	\framesubtitle{Trường hợp 4}
	\begin{twotrees}
	{
		\begin{forest}
			rb-tree, for tree={fit=band,l=0pt,s sep=0.6em}
			[,empty-node
				[B
					[A,label=left:x
						[$\alpha$,non-decorate-node]
						[$\beta$,non-decorate-node]
					]
					[D,red-node,label=right:s
						[C
							[$\gamma$,non-decorate-node]
							[$\delta$,non-decorate-node]
						]
						[E
							[$\varepsilon$,non-decorate-node]
							[$\zeta$,non-decorate-node]
						]
					]
				]
			]
		\end{forest}
	}
	{
		\begin{forest}
			rb-tree,for tree={fit=band,l=0pt,s sep=0.6em}
			[,empty-node
				[D,
					[B,red-node
						[A,label=left:x
							[$\alpha$,non-decorate-node]
							[$\beta$,non-decorate-node]
						]
						[C,label=below right:{new s}
							[$\gamma$,non-decorate-node]
							[$\delta$,non-decorate-node]
						]
					]
					[E,
						[$\varepsilon$,non-decorate-node]
						[$\zeta$,non-decorate-node]
					]
				]
			]
		\end{forest}
	}
	{2cm}
	{/}
	\end{twotrees}
\end{frame}

\begin{frame}
	\frametitle{Đánh giá sửa lỗi xóa trong cây đỏ-đen}
	Trong những trường hợp nào thì thì các thao tác sửa lỗi xóa thực hiện trong $O(1)$?
	\begin{itemize}
	\item Trường hợp 1
	\item Trường hợp 2
	\item \alert<3>{Trường hợp 4}
	\end{itemize}
\end{frame}