%!TeX program = lualatex
%!TeX root = main.tex

% --- CẤU TRÚC LỆNH MỚI ---
% \newcommand{\TenLenh}[Tổng số tham số][Giá trị mặc định của #1]{Nội dung}
\newcommand{\VeTamGiacGradient}[5][right]{
    % #1: [Tùy chọn] Vị trí (left hoặc right). Mặc định là "right".
    % #2: Tọa độ ĐỈNH (Tọa độ cũ là #1 nay thành #2)
    % #3: Chiều cao   (Tọa độ cũ là #2 nay thành #3)
    % #4: Độ dài đáy  (Tọa độ cũ là #3 nay thành #4)
    % #5: Text        (Tọa độ cũ là #4 nay thành #5)

    % 1. Vẽ tam giác (Dùng #2, #3, #4)
    \path[left color=white, right color=gray, draw=black] 
        (#2) coordinate (Dinh)
        -- ($(Dinh) + (-#4/2, -#3)$) 
        -- ($(Dinh) + (#4/2, -#3)$) 
        -- cycle;

    % 2. Xử lý vị trí Node dựa trên #1
    \ifthenelse{\equal{#1}{left}}{
        % --- NẾU LÀ LEFT ---
        % Vị trí x: trừ đi 1/2 đáy (sang trái)
        % anchor=east: Cạnh phải của chữ dính vào hình
        % xshift: Đẩy nhẹ ra xa hình một chút (dương) hoặc dính vào (âm) tùy bạn
        \node[anchor=east, xshift=1mm] at ($(Dinh) + (-#4/2, -#3/2)$) {\tiny #5};
    }{
        % --- NẾU LÀ RIGHT (Mặc định) ---
        % Vị trí x: cộng thêm 1/2 đáy (sang phải)
        % anchor=west: Cạnh trái của chữ dính vào hình
        \node[anchor=west, xshift=-1mm] at ($(Dinh) + (#4/2, -#3/2)$) {\tiny #5};
    }
}

% --- LỆNH VẼ NỬA TAM GIÁC (TAM GIÁC VUÔNG) ---
\newcommand{\VeNuaTamGiac}[5][right]{
    % #1: [left/right] Hướng vẽ và vị trí text (Mặc định: right)
    % #2: Tọa độ ĐỈNH
    % #3: Chiều cao
    % #4: Độ rộng đáy (của nửa tam giác)
    % #5: Text

    \ifthenelse{\equal{#1}{left}}{
        % =======================
        % TRƯỜNG HỢP: LEFT (Vẽ sang trái)
        % =======================
        % Gradient: Phải (trục) trắng -> Trái (cạnh huyền) xám
        \path[left color=gray, right color=white, draw=black] 
            (#2) coordinate (Dinh)
            -- ($(Dinh) + (0, -#3)$)    % Cạnh dọc (xuống dưới)
            -- ($(Dinh) + (-#4, -#3)$)  % Sang trái
            -- cycle;                   % Về đỉnh (tạo cạnh huyền)

        % Text nằm bên TRÁI cạnh huyền
        \node[anchor=east, xshift=-1mm] at ($(Dinh) + (-#4/2, -#3/2)$) {\tiny #5};
    }{
        % =======================
        % TRƯỜNG HỢP: RIGHT (Vẽ sang phải - Mặc định)
        % =======================
        % Gradient: Trái (trục) trắng -> Phải (cạnh huyền) xám
        \path[left color=white, right color=gray, draw=black] 
            (#2) coordinate (Dinh)
            -- ($(Dinh) + (0, -#3)$)    % Cạnh dọc
            -- ($(Dinh) + (#4, -#3)$)   % Sang phải
            -- cycle;

        % Text nằm bên PHẢI cạnh huyền
        \node[anchor=west, xshift=1mm] at ($(Dinh) + (#4/2, -#3/2)$) {\tiny #5};
    }
}

\section{Cây AVL}

\subsection{Giới thiệu}

\begin{frame}
    \frametitle{Thành phần của nút trong cây AVL}
    Nút trong cây AVL gồm các thuộc tính sau:
    \begin{description}
        \item[key] Giá trị của nút.
        \item[left] Nút con bên trái.
        \item[right] Nút con bên phải.
        \item[height] Chiều cao của nút.
    \end{description}
\end{frame}


\begin{frame}
    \frametitle{Định nghĩa độ cao}
    Độ cao của một nút là độ dài đường đi dài nhất từ nút đó đến lá
    \[
    h(x) = \begin{cases} 
        0 & \text{nếu $x = nil$} \\
        1 + \max(h(n.left), h(n.right)) & \text{nếu $x \neq nil$}
    \end{cases}
    \]
\end{frame}


\begin{frame}
    \frametitle{Định nghĩa hệ số cân bằng}
    Hệ số cân bằng của một nút được tính bằng
    \[
    bf(x) = h(n.right) - h(n.left)
    \]
\end{frame}


\begin{frame}
    \frametitle{Định nghĩa cây AVL}
    Cây AVL được đặt tên theo Adelson-Velsky và Landis, là cây nhị phân tìm kiếm thỏa mãn độ lớn hệ số cân bằng của mọi nút không quá một
\end{frame}

\begin{frame}
    \frametitle{Nhận xét}
    \begin{itemize}
        \item Mọi cây con của cây AVL cũng là cây AVL.
    \end{itemize}
\end{frame}


\begin{frame}
    \frametitle{Ví dụ về cây AVL}
    \centering
    \begin{forest}
        for tree={s sep= 3mm},
        tree
        [30,label=above right:{\scriptsize -1}
            [20,label=above left:{\scriptsize 0}
                [10,label=above left:{\scriptsize -1}
                    [5,label=above left:{\scriptsize 0}
                        [,nil-node]
                        [,nil-node]
                    ]
                    [,nil-node]
                ]
                [25,label=above left:{\scriptsize 1}
                    [,nil-node]
                    [35,label=above right:{\scriptsize 0}
                        [,nil-node]
                        [,nil-node]
                    ]
                ]
            ]
            [45,label=above right:{\scriptsize 0}
                [40,label=above right:{\scriptsize 0}
                    [,nil-node]
                    [,nil-node]
                ]
                [50,label=above right:{\scriptsize 0}
                    [,nil-node]
                    [,nil-node]
                ]
            ]
        ]
    \end{forest}
\end{frame}


\subsection{Các thao tác trên cây}
\begin{frame}
\frametitle{Phép quay}
\framesubtitle{Minh họa}
    \centering
    \begin{twotrees}
    {
        \begin{forest}
            tree,
            for tree={fit=tight,font={\scriptsize},s sep=2em,l=5mm}
            [,empty-node
                [y
                    [x
                        [$\alpha$,non-decorate-node]
                        [$\beta$,non-decorate-node]
                    ]
                    [$\gamma$,non-decorate-node]
                ]
            ]
        \end{forest}
    }
    {
        \begin{forest}
            tree,
            for tree={fit=tight,font=\scriptsize,s sep=2em,l=5mm}
            [,empty-node
                [x
                    [$\alpha$,non-decorate-node]
                    [y
                        [$\beta$,non-decorate-node]
                        [$\gamma$,non-decorate-node]
                    ]
                ]
            ]
        \end{forest}
    }
    {2cm}
    {Xoay phải / Xoay trái}
    \end{twotrees}
\end{frame}


\begin{frame}
    \frametitle{Xử lý mất cân bằng trong cây AVL}
    \framesubtitle{Trường hợp LL}
    \centering
    \begin{twotrees}
    {
        % Cây trước khi xoay
        \begin{forest}
            tree,
            for tree={l sep=0pt,s sep=1.5em}
            [,empty-node
                [D,s sep=2em,label=above left:{\scriptsize -2}
                    [B,label=above left:{\scriptsize -1}
                        [A,name=A]
                        [C,name=C]
                    ]
                    [E,name=E]
                ]
            ]
            \VeTamGiacGradient{A.south}{1cm}{0.3cm}{$h+1$}
            \VeTamGiacGradient{C.south}{1cm}{0.3cm}{$h$}
            \VeTamGiacGradient{E.south}{1cm}{0.3cm}{$h$}
        \end{forest}
    }
    {
        % Cây sau khi xoay phải
        \begin{forest}
            tree,
            for tree={l sep=0pt, s sep=1.5em}
            [,empty-node
                [B,s sep=2em,label=above right:{\scriptsize 0}
                    [A,name=A]
                    [D,label=above right:{\scriptsize 0}
                        [C,name=C]
                        [E,name=E]
                    ]
                ]
            ]
            \VeTamGiacGradient[left]{A.south}{1cm}{0.3cm}{$h+1$}
            \VeTamGiacGradient{C.south}{1cm}{0.3cm}{$h$}
            \VeTamGiacGradient{E.south}{1cm}{0.3cm}{$h$}
        \end{forest}
    }
    {2cm} % Khoảng cách giữa 2 cây
    {/}
    \end{twotrees}
\end{frame}


\begin{frame}
    \frametitle{Xử lý mất cân bằng trong cây AVL}
    \framesubtitle{Trường hợp LR}
    \centering
    \begin{threetrees}
    {
        % Cây trước khi xoay
        \begin{forest}
            tree,
            for tree={l sep=0pt,s sep=1.5em}
            [,empty-node
                [D,s sep=2em,label=above left:{\scriptsize -2}
                    [B,label=above left:{\scriptsize 1}
                        [A,name=A]
                        [C,name=C]
                    ]
                    [E,name=E]
                ]
            ]
            \VeTamGiacGradient{A.south}{1cm}{0.3cm}{$h$}
            \VeTamGiacGradient{C.south}{1cm}{0.3cm}{$h+1$}
            \VeTamGiacGradient{E.south}{1cm}{0.3cm}{$h$}
        \end{forest}
    }
    {
        % Cây sau khi xoay trái 
        \begin{forest}
            tree,
            for tree={l sep=0pt, s sep=0.5em}
            [,empty-node
                [D,label=above left:{\scriptsize -2}
                    [C,name=C,label=above left:{\scriptsize -1}
                        [B,name=B
                            [A,name=A]
                            [,phantom]
                        ]
                        [,phantom]
                    ]
                    [E, name=E
                        [,phantom]
                        [,phantom]
                    ]
                ]
            ]
            \VeTamGiacGradient{A.south}{1cm}{0.3cm}{$h$}
            \VeTamGiacGradient{E.south}{1cm}{0.3cm}{$h$}
            \VeNuaTamGiac{B.south east}{1cm}{0.3cm}{$h+1$}
            \VeNuaTamGiac{C.south east}{1cm}{0.3cm}{$h+1$}
        \end{forest}
    }
    {
        % Cây sau khi xoay phải 
        \begin{forest}
            tree,
            for tree={l sep=0pt, s sep=0.5em}
            [,empty-node
                [C,s sep=1.5em,label=above right:{\scriptsize 0}
                    [B,name=B,label=above left:{\scriptsize 0}
                        [A,name=A]
                        [,phantom]
                    ]
                    [D,name=D,label=above right:{\scriptsize 0}
                        [,phantom]
                        [E,name=E]
                    ]
                ]
            ]
            \VeTamGiacGradient{A.south}{1cm}{0.3cm}{$h$}
            \VeTamGiacGradient{E.south}{1cm}{0.3cm}{$h$}
            \VeNuaTamGiac{B.south east}{1cm}{0.3cm}{$h+1$}
            \VeNuaTamGiac[left]{D.south west}{1cm}{0.3cm}{}
        \end{forest}
    }
    {0.5cm} % Khoảng cách giữa 2 cây
    {/}
    \end{threetrees}
\end{frame}


\begin{frame}
    \frametitle{Xử lý mất cân bằng trong cây AVL}
    \framesubtitle{Trường hợp RR}
    \centering
    \begin{twotrees}
    {
        % --- TRƯỚC KHI XOAY (Lệch phải) ---
        \begin{forest}
            tree,
            for tree={l sep=0pt,s sep=1.5em}
            [,empty-node
                [B,s sep=2em,label=above right:{\scriptsize 2}
                    [A,name=A]
                    [D,label=above right:{\scriptsize 1}
                        [C,name=C]
                        [E,name=E]
                    ]
                ]
            ]
            % A và C cao h, E cao h+1 => Lệch về phía E (Phải-Phải)
            \VeTamGiacGradient[left]{A.south}{1cm}{0.3cm}{$h$}
            \VeTamGiacGradient[left]{C.south}{1cm}{0.3cm}{$h$}
            \VeTamGiacGradient{E.south}{1cm}{0.3cm}{$h+1$}
        \end{forest}
    }
    {
        % --- SAU KHI XOAY TRÁI ---
        \begin{forest}
            tree,
            for tree={l sep=0pt, s sep=1.5em}
            [,empty-node
                [D,s sep=2em,label=above left:{\scriptsize 0}
                    [B,label=above left:{\scriptsize 0}
                        [A,name=A]
                        [C,name=C]
                    ]
                    [E,name=E]
                ]
            ]
            % Cây cân bằng
            \VeTamGiacGradient[left]{A.south}{1cm}{0.3cm}{$h$}
            \VeTamGiacGradient{C.south}{1cm}{0.3cm}{$h$}
            \VeTamGiacGradient{E.south}{1cm}{0.3cm}{$h+1$}
        \end{forest}
    }
    {2cm} {/} % Mũi tên hoặc dấu phân cách
    \end{twotrees}
\end{frame}


\begin{frame}
    \frametitle{Xử lý mất cân bằng trong cây AVL}
    \framesubtitle{Trường hợp RL}
    \centering
    \begin{threetrees}
    {
        % --- CÂY 1: TRƯỚC KHI XOAY (Mất cân bằng RL) ---
        % Gốc A lệch phải về C, C lại lệch trái về B
        \begin{forest}
            tree,
            for tree={l sep=0pt,s sep=1.5em}
            [,empty-node
                [A,s sep=2em,label=above right:{\scriptsize 2}
                    [D,name=D] % Cây con trái A (h)
                    [C,label=above right:{\scriptsize -1}
                        [B,name=B]   % Cây con trái C (Gây mất cân bằng - h+1)
                        [E,name=E]   % Cây con phải C (h)
                    ]
                ]
            ]
            % Vẽ tam giác
            \VeTamGiacGradient[left]{D.south}{1cm}{0.3cm}{$h$}
            \VeTamGiacGradient[left]{B.south}{1cm}{0.3cm}{$h+1$}
            \VeTamGiacGradient{E.south}{1cm}{0.3cm}{$h$}
        \end{forest}
    }
    {
        % --- CÂY 2: SAU KHI XOAY PHẢI TẠI C ---
        % B được đẩy lên làm con phải của A. C tụt xuống làm con phải của B.
        \begin{forest}
            tree,
            for tree={l sep=0pt, s sep=0.5em}
            [,empty-node
                [A,label=above right:{\scriptsize 2}
                    [D, name=D
                        [,phantom]
                        [,phantom]
                    ]
                    [B, name=B,label=above right:{\scriptsize 1}
                        [,phantom]
                        [C, name=C
                            [,phantom]
                            [E,name=E]
                        ]
                    ]
                ]
            ]
            \VeTamGiacGradient[left]{D.south}{1cm}{0.3cm}{$h$}
            \VeTamGiacGradient{E.south}{1cm}{0.3cm}{$h$}
            \VeNuaTamGiac[left]{B.south west}{1cm}{0.3cm}{$h+1$} % Phần trái của B
            \VeNuaTamGiac[left]{C.south west}{1cm}{0.3cm}{$h+1$} % Phần phải của B
        \end{forest}
    }
    {
        % --- CÂY 3: SAU KHI XOAY TRÁI TẠI A (CÂN BẰNG) ---
        % B lên làm gốc. A xuống làm con trái. C là con phải.
        \begin{forest}
            tree,
            for tree={l sep=0pt, s sep=0.5em}
            [,empty-node
                [B,s sep=1.5em,label=above right:{\scriptsize 0}
                    [A,name=A,label=above left:{\scriptsize 0}
                        [D,name=D]
                        [,phantom]
                    ]
                    [C,name=C,label=above right:{\scriptsize 0}
                        [,phantom]
                        [E,name=E]
                    ]
                ]
            ]
            \VeTamGiacGradient[left]{D.south}{1cm}{0.3cm}{$h$}
            \VeTamGiacGradient{E.south}{1cm}{0.3cm}{$h$}
            % Các phần tách ra của B được chia đều cho A và C
            \VeNuaTamGiac{A.south east}{1cm}{0.3cm}{$h+1$}        % Phần trái của B (về A)
            \VeNuaTamGiac[left]{C.south west}{1cm}{0.3cm}{}  % Phần phải của B (về C)
        \end{forest}
    }
    {0.5cm} % Khoảng cách
    {/} % Text mũi tên
    \end{threetrees}
\end{frame}


\begin{frame}
    \frametitle{Phép chèn}
    \begin{algorithm}[H]
    \small
    \SetAlgoLined
    1. Thực hiện phép chèn của BST\\
    2. Cập nhật chiều cao của nút hiện tại\\
    3. Tính hệ số cân bằng của nút hiện tại\\
    4. Xử lí các trường hợp cân bằng nếu nút mất cân bằng\\
    5. Lặp lại bước 2-4 cho đến nút gốc\\
    \caption{Phép chèn nút cây AVL}
    \end{algorithm}

\end{frame}


\begin{frame}
    \frametitle{Phép xóa}
    \begin{algorithm}[H]
    \small
    \SetAlgoLined
    1. Thực hiện phép xóa của BST\\
    2. Cập nhật chiều cao của nút hiện tại\\
    3. Tính hệ số cân bằng của nút hiện tại\\
    4. Xử lí các trường hợp cân bằng (LL, LR, RR, RL) nếu mất cân bằng\\
    5. Lặp lại bước 2-4 đi lên đến nút gốc\\
    \caption{Phép xóa nút cây AVL}
    \end{algorithm}
\end{frame}
