%!TeX root = main.tex
%!TeX program = lualatex 
\section{Đặt vấn đề}

\subsection{Tổng quan về BST}
% BST 
\begin{frame}
    \frametitle{\insertsection}
    \framesubtitle{Tổng quan về Binary Search Tree (BST)}

    \begin{columns}
    %---- Left column ----
    \column{0.5\textwidth}
    \begin{block}{BST là gì?}
        Một cấu trúc cây nhị phân trong đó:
        \begin{itemize}
            \item Mỗi node có giá trị.
            \item Node trái chứa giá trị \textbf{nhỏ hơn}.
            \item Node phải chứa giá trị \textbf{lớn hơn}.
        \end{itemize}
    \end{block}

    %---- Right column ----
    \column{0.5\textwidth}
    \centering
    \begin{forest}
    tree,for tree={font=\normalfont,fit=band,minimum size=2em}
    [8
    	[4
    		[2]
    		[6]
    	]
    	[12
    		[10]
    		[14]
	]
  ]
    \end{forest}

    \end{columns}
\end{frame}

%BST Insert 

\begin{frame}
    \frametitle{\insertsection}
    \framesubtitle{Chèn và Tìm kiếm trên BST}

    \begin{columns}
    \column{0.55\textwidth}

     \begin{block}{Tìm kiếm}
        \begin{itemize}
            \item So sánh từ gốc đến lá.
            \item Sang trái nếu nhỏ hơn, sang phải nếu lớn hơn.
            \item Tìm thấy trả về node, không thì kết thúc.
        \end{itemize}
    \end{block}

    \begin{block}{Chèn}
        \begin{itemize}
            \item So sánh từ gốc đến lá.
            \item Sang trái nếu nhỏ hơn, sang phải nếu lớn hơn.
            \item Đến node lá thì thêm node mới vào với cha là node lá.
        \end{itemize}
    \end{block}

    \column{0.45\textwidth}
    \centering
    \begin{forest}
    tree,for tree={font=\normalfont,fit=band,minimum size=2em}
    [8
    	[4
    		[2]
    		[6]
    	]
    	[12
    		[10]
    		[14]
	]
  ]
    \end{forest}
    \end{columns}
\end{frame}

% BST Delete 
\begin{frame}
    \frametitle{\insertsection}
    \framesubtitle{Xóa trong BST}

    \begin{columns}
    \column{0.55\textwidth}
    \begin{block}{Có 3 trường hợp xóa}
        \begin{itemize}
            \item \textbf{Node lá}: xóa trực tiếp.
            \item \textbf{Node 1 con}: thay node bằng con đó.
            \item \textbf{Node 2 con}: 
            \begin{itemize}
                \item Tìm node kế tiếp (nhỏ nhất cây con phải).
                \item Thay giá trị node cần xóa bằng giá trị đó.
                \item Xóa node kế tiếp (chỉ còn 0 hoặc 1 con).
            \end{itemize}
        \end{itemize}
    \end{block}
    \column{0.45\textwidth}
    \centering
    \begin{forest}
    tree,for tree={font=\normalfont,fit=band,minimum size=2em}
	[8
		[4
			[6]
		]
		[12
			[10]
			[14
				[13]
				[15
					[16]
				]
			]
		]
	]
	\end{forest}
    \end{columns}
\end{frame}

% BST Property

\begin{frame}
    \frametitle{\insertsection}
    \framesubtitle{Đặc điểm của Binary Search Tree}

    \begin{columns}
    %---- Left column ----
    \column{0.5\textwidth}
    \begin{block}{Đặc điểm chính}
        \begin{itemize}
            \item Tìm kiếm trung bình: $O(\log n)$.
            \item Chèn và xoá tương đối đơn giản.
            \item Cấu trúc linh hoạt, dễ cài đặt.
        \end{itemize}
    \end{block}

    %---- Right column ----
    \column{0.5\textwidth}
    \centering
    \begin{forest}
    tree,for tree={font=\normalfont,fit=band,minimum size=2em}
    [8
    	[4
    		[2]
    		[6]
    	]
    	[12
    		[10]
    		[14]
	]
  ]
    \end{forest}
    \end{columns}
\end{frame}

\subsection{Vấn đề về cân bằng}
\begin{frame}
    \frametitle{\insertsection}
    \framesubtitle{Tại sao cần cây tự cân bằng?}

    \begin{columns}
    %===== LEFT COLUMN =====
    \column{0.55\textwidth}
    \begin{block}{Vấn đề của cây nhị phân tìm kiếm (BST)}
        BST thông thường rất phụ thuộc vào \textbf{thứ tự chèn}. 
        Nếu dữ liệu được chèn theo thứ tự tăng/giảm dần:
        \begin{itemize}
            \item Cây trở thành dạng \alert{danh sách liên kết}.
            \item Độ cao đạt $O(n)$.
            \item Tìm kiếm, chèn, xoá đều có độ phức tạp $O(n)$.
        \end{itemize}
        Điều này không phù hợp cho các hệ thống cần hiệu năng ổn định.
    \end{block}

    %===== RIGHT COLUMN =====
    \column{0.45\textwidth}
    \centering
    \begin{forest}
    tree,for tree={font=\normalfont,fit=band,minimum size=2em}
    [1
        [,phantom]        % left child (empty)
        [2
            [,phantom]
            [3
                [,phantom]
                [4
                    [,phantom]
                    [5]
                ]
            ]
        ]
    ]
    \end{forest}
    \end{columns}

\end{frame}


\begin{frame}
    \frametitle{\insertsection}
    \framesubtitle{Tại sao cần cây tự cân bằng?}
    \begin{alertblock}{Nhu cầu thực tế}
        Cần một cấu trúc dữ liệu đảm bảo:
        \begin{itemize}
            \item Độ cao luôn xấp xỉ $O(\log n)$.
            \item Hiệu năng ổn định bất kể thứ tự đầu vào.
            \item Tối ưu cho tìm kiếm hoặc cập nhật tùy tình huống.
        \end{itemize}
    \end{alertblock}
\end{frame}

\subsection{Bài toán}
\begin{frame}
    \frametitle{\insertsection}
    \framesubtitle{Bài toán đặt ra}

    \begin{block}{Bài toán}
        Thiết kế một cây nhị phân tìm kiếm \textbf{tự cân bằng} sao cho:
        \begin{enumerate}
            \item Duy trì được độ cao $O(\log n)$.
            \item Cho phép chèn, xoá, tìm kiếm hiệu quả.
            \item Không phụ thuộc vào thứ tự dữ liệu đầu vào.
        \end{enumerate}
    \end{block}

    \begin{block}{Hai hướng giải khác nhau}
        Từ nhu cầu đó sinh ra hai cây điển hình:
        \begin{itemize}
            \item \textbf{Cây AVL}: Cân bằng \textit{nghiêm ngặt} để tối ưu tốc độ tìm kiếm.
            \item \textbf{Cây Đỏ-Đen}: Cân bằng \textit{vừa đủ} để giảm số phép xoay khi chèn/xoá.
        \end{itemize}
        Mỗi loại cây phù hợp với các ứng dụng khác nhau.
    \end{block}
\end{frame}

\begin{frame}
    \frametitle{\insertsection}
    \framesubtitle{Tại sao tồn tại cả AVL và Red-Black?}

    \begin{block}{Hai nhu cầu tối ưu khác nhau}
        \begin{itemize}
            \item Khi hệ thống cần \alert{tìm kiếm rất nhanh}, dữ liệu ít thay đổi → chọn \textbf{AVL}.
            \item Khi hệ thống cần \alert{chèn/xoá thường xuyên}, cập nhật liên tục → chọn \textbf{Cây Đỏ Đen}.
        \end{itemize}
    \end{block}
\end{frame}
